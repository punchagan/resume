%-------------------------------------------------------------------------------
%	SECTION TITLE
%-------------------------------------------------------------------------------
\cvsection{Work Experience}


%-------------------------------------------------------------------------------
%	CONTENT
%-------------------------------------------------------------------------------
\begin{cventries}

%---------------------------------------------------------
  \cventry
      {Senior Software Developer} % Job title
      {Akvo Foundation \externallink{https://akvo.org}} % Organization
      {Remote} % Location
      {Sep. 2016 - Present} % Date(s)
      {
        \begin{cvitems} % Description(s) of tasks/responsibilities
        \item {Primarily worked on \href{https://github.com/akvo/akvo-rsr}{RSR} --- a Django based tool for organisations working in the development sector}
        \item {Lead back-end developer for RSR, for the past 2+ years, helping the team design and implement new features, while also getting rid of tech-debt and helping keep the team productive}
        \item {Over the years, contributed a bunch of few performance improvements and general code quality improvements, apart from adding a bunch of new features and fixing bugs}
        \item {Collaborated on building the \href{https://digital.gpmarinelitter.org/}{Marine Litter Digital Platform} --- a site that aggregates resources to guide action on marine litter. It uses Clojure in the back-end, Postgres as the database and React on the front-end}
        \end{cvitems}
      }

%---------------------------------------------------------
  \cventry
    {Machine Learning Engineer} % Job title
    {Infilect Tech. Pvt. Ltd. \externallink{https://infilect.com}} % Organization
    {Bangalore, India} % Location
    {Jul. 2015 - Mar. 2016} % Date(s)
    {
      \begin{cvitems} % Description(s) of tasks/responsibilities
      \item{Collaborated on a fashion assistant leveraging deep neural networks for image based search and retrieval. Specifically worked on trying different state-of-the-art techniques published by different groups to fine-tune the search results}
        \item{Collaborated on designing and implementing the initial data collection platform for the products listing for the fashion assistant}
        \item{Trained and deployed deep learning models as proof-of-concepts for different potiential enterprise clients}
      \end{cvitems}
    }

%---------------------------------------------------------
  \cventry
    {Software Engineer} % Job title
    {Enthought Inc. \externallink{https://www.enthought.com}} % Organization
    {Mumbai, India} % Location
    {Sep. 2011 - May 2014} % Date(s)
    {
      \begin{cvitems} % Description(s) of tasks/responsibilities
        \item {Primarily worked on \href{https://web.archive.org/web/20170621202502/https://www.enthought.com/products/canopy/}{Canopy} --- a Qt based development environment for Scientists/Engineers. The product has been discontinued a few years later, when Anaconda started to emerge as a clear winner}
        \item {Made small contributions and improvements to different pieces of the Enthought Tool Suite and the IPython Qt console and notebook}
        \item {Acquired a \href{https://rawgit.com/punchagan/340e1350fdfc766c6599/raw/ca1f5fe9bfc1cc503cd8a524e350bd29e8f5f33d/month-with-martin.html}{taste} for good software design, working along-side some very experienced programmers}
      \end{cvitems}
    }

%---------------------------------------------------------
  \cventry
    {Research Associate} % Job title
    {FOSSEE, IIT Bombay \externallink{https://fossee.in}} % Organization
    {Mumbai, India} % Location
    {Jul. 2009 - Jul. 2011} % Date(s)
    {
      \begin{cvitems} % Description(s) of tasks/responsibilities
        \item {Designed and developed material for a course on \href{https://github.com/FOSSEE/sees}{Software Engineering for Engineers and Scientists}}
        \item {Conducted a series of workshops across the country in conferences and colleges to train students and teachers in using Python (primarily numpy, scipy and matplotlib) for their Scientific Computing needs}
      \end{cvitems}
    }

%---------------------------------------------------------
\end{cventries}
